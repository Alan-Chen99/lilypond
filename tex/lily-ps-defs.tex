%
% WARNING: don't leave blank lines in the PS-code; they are
% transformed into \par 
%

%
% header info (macros/defs, etc) should go into a \special{! ... }, 
% note the ! sign.  See dvips.info for details.
%

% Use of 
% /foo { operatorname } bind def
%
% ``compiles'' operatorname binding in the body of foo, making
% the code faster, and more reliable (less flexible)

%must come before PSTeXDimen
\special{!
% PS helper: convert (0.2pt) to the token 0.2
/settexdimen
{
        /thestring exch def
        thestring 0 thestring length 2 sub
        getinterval
        token
        pop exch pop 
} def
%
/deftexdimen
{
        settexdimen
        def     
} def
}
% transplant a TeX dimension into the PS output.
\def\PSsetTeXdimen#1{\expandafter\special{! /#1 (\the\csname #1\endcsname) deftexdimen}}
% must come before lily.ps
\PSsetTeXdimen{stafflinethickness}
\PSsetTeXdimen{staffheight}
{%
   \def\par{ }%         %Ugh.  Don't try this at home, kids!
   % neat file-include trick by Piet van Oostrum <piet@cs.uu.nl>
   \newread\defin 
   \newtoks\toksfiledefi\newtoks\toksfiledefii 
   \def\ifnot#1{#1\else\expandafter\expandafter\fi\iffalse\iftrue\fi} 
   \def\filedef#1#2{%#1=command name, #2=file name 
           \openin\defin=#2\relax\def#1{} 
           \ifeof\defin
                   \message{***************************************}
                   \message{lily-ps-defs.tex: can't open `#2'}
                   \message{***************************************}
                   \end
           \fi 
           \loop\ifnot{\ifeof\defin}\read\defin to\tempfiledef
           \toksfiledefi=\expandafter{#1}% 
           \toksfiledefii=\expandafter{\tempfiledef}% 
   %        \expandafter\special\expandafter{!\the\toksfiledefi\the\toksfiledefii}\repeat%
          \global\edef#1{\the\toksfiledefi\the\toksfiledefii}\repeat%
   }
   %
   % This seems a little backwards, but we don't want to include the PS
   % stuff too early
   %
   \filedef\includelilyps{lily.ps}%
   \expandafter\special{! \includelilyps}
}



\def\turnOnPostScript{%
        \PSsetTeXdimen{stafflinethickness}
        \PSsetTeXdimen{staffheight}
        
        % This sets CTM so that you get to the currentpoint
        % by executing a 0 0 moveto
        \def\embeddedps##1{%
                \special{ps: @beginspecial @setspecial ##1 @endspecial}       
        }
        %

        \special{! 
/interline \mudelapaperinterline0  def % ugh.  Only works x.yyyy floats 
stafflinethickness 1.2 mul /plet_t exch def
interline 3 div /bracket_b exch def
interline 2 mul /bracket_w exch def
stafflinethickness 2 mul /bracket_t exch def
interline 1.5 mul /bracket_v exch def
bracket_v /bracket_u exch def
50 /bracket_alpha exch def
staffheight 4 div /interline exch def
1 setlinecap}
}

\def\turnOnExperimentalFeatures{}

