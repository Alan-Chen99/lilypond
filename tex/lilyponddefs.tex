%%
%% include file for LilyPond
%%
%% this file defines various macros to accomodate lilypond output
%% 
%% let's not make par before endinput
%
% TeXbook ex 7.7
\def\ifundefined#1{\expandafter\ifx\csname#1\endcsname\relax}
%
% skip if included already
\def\SkipLilydefs{\endinput}
\ifundefined{EndLilyPondOutput}
        \def\EndLilyPondOutput{\csname bye\endcsname}
        \def\SkipLilydefs{}
\fi
\SkipLilydefs
%
\ifundefined{mudelacopyright}
        \def\mudelacopyright{\copyright\ \number\year}
\fi
\ifundefined{LilyIdString}
        \def\LilyIdString{Lily was here}
\fi
\ifundefined{documentclass}
        \input lilypond-plaintex
\else
        \input lilypond-latex
\fi

%%%%%%%%%%%%%%%%%%%%%%%%%%%%%%%%%%%%%%%%%%%%%%%%%%%%%%%%%%%%%%%%
% macros to shorten other definitions
%%%%%%%%%%%%%%%%%%%%%%%%%%%%%%%%%%%%%%%%%%%%%%%%%%%%%%%%%%%%%%%%
\def\musicmathdef#1#2{\def#1{\musicmathchar{#2}}}
\def\musicmathchar#1{\musicmathfont\char#1}

\def\topalign#1{\vbox to 0pt{#1\vss}}
\def\botalign#1{\vbox to 0pt{\vss #1}}

\def\centeralign#1{\hbox to 0pt{\hss#1\hss}}
\def\leftalign#1{\hbox to 0pt{#1\hss}}
\def\rightalign#1{\hbox to 0pt{\hss#1}}


%%%%%%%%%%%%%%%%%%%%%%%%%%%%%%%%%%%%%%%%%%%%%%%%
%% set up dimensions
\parindent=0pt
\newdimen\smallspace
\newdimen\interlinedist

\newdimen\stemthickness
\newcount\n                     %duh. meaningful identifiers.
\newdimen\staffheight
\newdimen\notewidth
\newdimen\noteheight
\newdimen\staffrulethickness
\newdimen\interstaffrule
\newdimen\dist

%%%%%%%%%%%%%%%%%%%%%%%%%%%%%%%%%%%%%%%%%%%%%%%%%%%%%%%%%%%%%%%%
% set fonts and primary dimensions
% ugh

\def\cmrtwenty{
        \font\smalltextfont=cmr8
        \font\textmusic=cmmi12
}

\def\cmrsixteen{
        \font\smalltextfont=cmr6
        \font\textmusic=cmmi10
        }
\def\cmrthirteen{
        \font\smalltextfont=cmr6
        \font\textmusic=cmmi9
        }
\def\cmreleven{
        \font\smalltextfont=cmr5
        \font\textmusic=cmmi8
        }

%%%%%%%
%
\def\musixtwentydefs{
      \twentyfonts
      \csname cmrtwenty\texsuffix\endcsname
      \musixcalc
}

\def\musixsixteendefs{
        \sixteenfonts
        \csname cmrsixteen\texsuffix\endcsname
        \musixcalc
        }
\def\musixtwentysixdefs{
      \csname cmrtwentysix\texsuffix\endcsname
          \twentysixfonts
         \musixcalc
        }
\def\musixthirteendefs{
        \thirteenfonts
        \csname cmrthirteen\texsuffix\endcsname
       \musixcalc
}
\def\musixelevendefs{
        \csname cmreleven\texsuffix\endcsname
        \elevenfonts
        \musixcalc
}

\def\textsharp{\raise.4ex\hbox{\textmusic\char"5D}}
\def\textnatural{\raise.4ex\hbox{\textmusic\char"5C}}
\def\textflat{\raise.2ex\hbox{\textmusic\char"5B}}


%%%%%%%%%%%%%%%%%%%%%%%%%%%%%%%%%%%%%%%%%%%%%%%%%%%%%%%%%%%%%%%%
% do derivative calcs

% this has to be synced with the font definition
\def\musixcalc{
        \staffheight=\mudelapaperbarsize pt

        % ugh.  Can extract ex dim from TFM
        \interlinedist=\staffheight 
        \divide\interlinedist by 4
        \notewidth=\mudelapapernotewidth pt

        \smallspace=.3\interlinedist
        \interstaffrule=\staffheight
        \divide\interstaffrule by 4
        \staffrulethickness=\mudelapaperrulethickness pt
        \stemthickness=\staffrulethickness
}



% stacked horizontal lines 
\def\interscoreline{\vskip 16pt}
\def\setdynamic#1{\dynfont #1}
\def\setfinger#1{\fingerfont #1}
\def\setnumber#1{\fetanummer #1}
\def\setmark#1{\markfont #1}

% big fat marks, if errors are detected.
\def\columnerrormark{\placebox{-5pt}{0pt}{\bf C!}}
\def\scorelineerrormark{\placebox{0pt}{-10pt}{\bf L!}}
\def\errormark{{\bf E!}}
\def\unknown{%
  %{\bf u} %FIXME
}

\input dyndefs
\input fetdefs



\def\emptybar{}

\def\defaultthinbar{\thinbar{\staffheight}}
\def\defaultthickbar{\thickbar{\staffheight}}
%? what-s wrong with rightalign?
\def\repeatstopbar{\rightalign{\repeatcolon\kern2\smallspace\defaultthinbar\kern\smallspace\defaultthickbar}}
\def\repeatstartbar{\hbox{\defaultthickbar\kern\smallspace\defaultthinbar\kern2\smallspace\repeatcolon}}
\def\repeatstopstart{\hbox{\repeatcolon\kern2\smallspace\defaultthickbar\kern\smallspace\defaultthickbar\kern2\smallspace\repeatcolon}}

%compatibility
%urg
\fetdef\repeatcolon{20}
\def\repeatbar#1{\repeatstopbar}
\def\startrepeat#1{\repeatstartbar}
\def\repeatbarstartrepeat#1{\repeatstopstart}

%%%%%%%%%%%%%%%%%%%%%%%%%%%%%%%%%%%%%%%%%%%%%%%%%
% parametric symbols
%%%%%%%%%%%%%%%%%%%%%%%%%%%%%%%%%%%%%%%%%%%%%%%%%

\def\doublebar#1{\hbox{\thinbar{#1}\hskip\smallspace\thinbar{#1}}}
\def\thinbar#1{\dimen0=#1%
        \vrule height .5\dimen0 depth .5\dimen0 width 1.6\staffrulethickness} % TODO parametric.
\def\thickbar#1{\dimen0=#1%
        \vrule height .5\dimen0 depth .5\dimen0 width 2\smallspace}
\def\maatstreep#1{\thinbar{#1}}
\def\startbar#1{\leftalign{\thickbar{#1}\kern\smallspace\thinbar{#1}}}
\def\finishbar#1{\rightalign{\thinbar{#1}\kern\smallspace\thickbar{#1}}}
\def\fatdoublebar#1{\hbox{\phantom{\repeatcolon\kern2\smallspace}\thickbar{#1}\kern\smallspace\thickbar{#1}}}



% ugh
% see e.g. input/test/beam-pos.ly
%
% something's wrong with the aligment; sometimes all symbols
% look to be placed a bit too high (there's an ugly fix for
% the staccato-dot in script.cc)
% but this varies from line to line: it seems that xdvi
% does some rounding; i can't really check this from screen on i
% 600x600 res.
%
\def\rulesym#1#2{\dimen0=#1%
        \vrule height .5\dimen0 depth .5\dimen0 width #2}
\def\shiftedrulesym#1#2{\dimen0=#1%
        \vrule height .7\dimen0 depth .3\dimen0 width #2}
\def\tinyrulesym#1#2{\dimen0=#1%
        \vrule height .1\dimen0 depth .1\dimen0 width #2}
%would be nice for checking alignment
\def\openrulesym#1#2{\dimen0=#1%
        \vbox to \dimen0{\vss%
        \hbox{\vrule height .1\dimen0 width #2}%
        \hbox{\vrule height .2\dimen0 width 0pt}%
        \hbox{\vrule height .4\dimen0 width #2}%
        \hbox{\vrule height .2\dimen0 width 0pt}%
        \hbox{\vrule height .1\dimen0 width #2}%
        \vss}}
%\let\rulesym\shiftedrulesym
%\let\rulesym\tinyrulesym
%\let\rulesym\openrulesym

% the interline symbol. Redefine to remove it.
\def\defaultlineseparator{\vbox{\mussepline\vskip -5pt\mussepline}}
\def\lineseparator{\defaultlineseparator}
\def\beauty{%
        \par\vskip 10pt plus 30pt minus 10pt\par
        \hskip -5pt\lineseparator
        \par\vskip 10pt plus 30pt minus 10pt\par
}


%%%%%%%%%%%%%%%%%%%%%%%%%%%%%%%%%
\def\postheader{}

%
% macros suck. \ifundefined{nolilyfooter} gives wierd errors from time
%        to time.
%

%
% Warning: the order is conceptually weird.  It says:

% if not defined (``nolilyfooter''):
%    do_footer ()
% else 
%    dont_do_footer ()
%
\ifx\csname nolilyfooter\endcsname\relax
        \message{[footer defined]}%
        \csname lilyfooter\texsuffix\endcsname%
\else
        \message{[footer empty]}
        \csname%
        nolilyfooter\texsuffix\endcsname
\fi
