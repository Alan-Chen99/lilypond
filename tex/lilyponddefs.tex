%%
%% include file for LilyPond
%%
%% this file defines various macros to accomodate lilypond output
%% 
%% let's not make par before endinput
%
% TeXbook ex 7.7
\def\ifundefined#1{\expandafter\ifx\csname#1\endcsname\relax}
%
% skip if included already
\def\SkipLilydefs{\endinput}
\ifundefined{EndLilyPondOutput}
        \def\EndLilyPondOutput{\csname bye\endcsname}
        \def\SkipLilydefs{}
\fi
\SkipLilydefs
%
\ifundefined{mudelacopyright}
        \def\mudelacopyright{\copyright\ \number\year}
\fi
\ifundefined{LilyIdString}
        \def\LilyIdString{Lily was here}
\fi

%%%%%%%%%%%%%%%%%%%%%%%%%%%%%%%%%%%%%%%%%%%%%%%%%%%%%%%%%%%%%%%%
% macros to shorten other definitions
%%%%%%%%%%%%%%%%%%%%%%%%%%%%%%%%%%%%%%%%%%%%%%%%%%%%%%%%%%%%%%%%
\def\musicdef#1#2{\def#1{\musicchar{#2}}}
\def\musicchar#1{\musicfnt\char#1}
\def\rationalmultiply#1*#2/#3{\multiply #1 by #2 \divide #1 by #3}


\def\maccentraise#1#2{\dimen0=\noteheight
        \rationalmultiply\dimen0*#2%
        \raise\dimen0\hbox{#1}}
\def\maccentdef#1#2#3{\def#1{\maccentraise{\musicchar{#2}}{#3}}}
\def\vertcenter#1{\vbox to 0pt{\vss #1\vss}}

\def\musicmathdef#1#2{\def#1{\musicmathchar{#2}}}
\def\musicmathchar#1{\musicmathfont\char#1}

\def\topalign#1{\vbox to 0pt{#1\vss}}
\def\botalign#1{\vbox to 0pt{\vss #1}}

\def\centeralign#1{\hbox to 0pt{\hss#1\hss}}
\def\leftalign#1{\hbox to 0pt{#1\hss}}
\def\rightalign#1{\hbox to 0pt{\hss#1}}


%%%%%%%%%%%%%%%%%%%%%%%%%%%%%%%%%%%%%%%%%%%%%%%%
%% set up dimensions
\parindent=0pt
\newdimen\smallspace
\newdimen\interlinedist
\newdimen\stemthickness
\newcount\n
\newdimen\staffheight
\newdimen\notewidth
\newdimen\noteheight
\newdimen\staffrulethickness
\newdimen\interstaffrule
\newdimen\dist

%%%%%%%%%%%%%%%%%%%%%%%%%%%%%%%%%%%%%%%%%%%%%%%%%%%%%%%%%%%%%%%%
% set fonts and primary dimensions
% ugh
\def\musixtwentydefs{
        \twentyfonts
        \musixcalc
}

\def\cmrtwenty{
        \font\meterfont=cmbx15
        \font\italicfont=cmti10 scaled \magstep1
        \font\musicmathfont=cmsy10
        \font\normaltextfont=cmr10 %\textfont is a primitive
        \font\smalltextfont=cmr8
        \font\boldfont=cmbx10
        \font\textmusic=cmmi12
        \font\largefont=cmbx12
}
\def\cmrsixteen{
        \font\smalltextfont=cmr6
        \font\normaltextfont=cmr8 %\textfont is a primitive
        \font\meterfont=cmbx12
        \font\italicfont=cmti9
        \font\textmusic=cmmi10
        \font\boldfont=cmbx8
        \font\largefont=cmbx10
}
\def\cmreleven{
        \font\smalltextfont=cmr5
        \font\normaltextfont=cmr6
        \font\meterfont=cmbx8
        \font\italicfont=cmti6
        \font\textmusic=cmmi8
        \font\boldfont=cmbx6
        \font\largefont=cmbx8
}

\def\cmrthirteen{
        \font\smalltextfont=cmr6
        \font\normaltextfont=cmr7
        \font\meterfont=cmbx9
        \font\italicfont=cmti7
        \font\textmusic=cmmi9
        \font\boldfont=cmbx7
        \font\largefont=cmbx9
}
\def\musixsixteendefs{
        \sixteenfonts
        \musixcalc
}
\def\musixtwentysixdefs{
        \twentysixfonts
        \musixcalc
}
\def\musixthirteendefs{
        \thirteenfonts
        \musixcalc
}
\def\musixelevendefs{
        \elevendefs
        \musixcalc
}

\def\textsharp{\raise.4ex\hbox{\textmusic\char"5D}}
\def\textnatural{\raise.4ex\hbox{\textmusic\char"5C}}
\def\textflat{\raise.2ex\hbox{\textmusic\char"5B}}




%%%%%%%%%%%%%%%%%%%%%%%%%%%%%%%%%%%%%%%%%%%%%%%%%%%%%%%%%%%%%%%%
% do derivative calcs

% this has to be synced with the font definition
\def\musixcalc{
        \staffheight=\mudelapaperbarsize pt
        \interlinedist=\staffheight
        \divide\interlinedist by 4
        \notewidth=\mudelapapernotewidth pt

        \smallspace=.3\interlinedist
        \interstaffrule=\staffheight
        \divide\interstaffrule by 4
        \staffrulethickness=\mudelapaperrulethickness pt
        \stemthickness=\staffrulethickness
}

% stacked numbers; may be overruled in fetdefs
\def\generalmeter#1#2{\vbox to 0pt{\vss\hbox{\meterfont
         #1}\nointerlineskip
         \hbox{\meterfont #2}\vss}}

% stacked horizontal lines 

\input dyndefs
\input fetdefs



\def\emptybar{}

\def\defaultthinbar{\thinbar{\staffheight}}
\def\defaultthickbar{\thickbar{\staffheight}}
%? what-s wrong with rightalign?
\def\repeatstopbar{\hss\rightalign{\repeatcolon\hskip2\smallspace\defaultthinbar\hskip\smallspace\defaultthickbar}}
\def\repeatstartbar{\hbox{\defaultthickbar\kern\smallspace\defaultthinbar\kern2\smallspace\repeatcolon}}
\def\repeatstopstart{\hbox{\repeatcolon\kern2\smallspace\defaultthickbar\kern\smallspace\defaultthickbar\kern2\smallspace\repeatcolon}}

%compatibility
\def\repeatbar{\repeatstopbar}
\def\startrepeat{\repeatstartbar}
\def\repeatbarstartrepeat{\repeatstopstart}

\def\vruler#1{{%
        \def\wid{\dimen0}%
        \def\inc{\dimen1}%
        \wid=#1pt
        \inc=\wid
        \divide\inc by #1
        \divide\wid by 2
        \here=-\wid
        \loop\ifdim\here<\wid\advance\here by\inc
                \hbox to0pt{\vbox to0pt{\vss\hrule width2pt height 0.05pt\kern\here}\hss}%
        \repeat%
}}
\def\hruler#1#2{\hbox{%
        \def\wid{\dimen0}%
        \def\here{\dimen3}%
        \wid=#1pt
        \divide\wid by 2
        \here=-\wid
        \loop\ifdim\here<\wid\advance\here by #2
                \hbox to0pt{\kern\here\vrule width0.05pt height 1pt depth 1pt\hss}%
        \repeat%
}}


%%%%%%%%%%%%%%%%%%%%%%%%%%%%%%%%%%%%%%%%%%%%%%%%%
% parametric symbols
%%%%%%%%%%%%%%%%%%%%%%%%%%%%%%%%%%%%%%%%%%%%%%%%%

\def\doublebar#1{\hbox{\thinbar{#1}\hskip\smallspace\thinbar{#1}}}
\def\thinbar#1{\dimen0=#1%
        \vrule height .5\dimen0 depth .5\dimen0 width 1.6\staffrulethickness} % TODO parametric.
\def\thickbar#1{\dimen0=#1%
        \vrule height .5\dimen0 depth .5\dimen0 width 2\smallspace}
\def\maatstreep#1{\thinbar{#1}}
\def\startbar#1{\leftalign{\thickbar{#1}\kern\smallspace\thinbar{#1}}}
\def\finishbar#1{\rightalign{\thinbar{#1}\kern\smallspace\thickbar{#1}}}


\def\stem#1#2{\hbox{\kern -.5\stemthickness
        \vrule width\stemthickness height#2 depth-#1}}

\def\placebox#1#2#3{%
        \botalign{\hbox{\raise #1\leftalign{\kern #2{}#3}}}%
}


\def\rulesym#1#2{\dimen0=#1%
        \vrule height .5\dimen0 depth .5\dimen0 width #2}
\def\settext#1{\normaltextfont #1}
\def\setitalic#1{\italicfont #1}
\def\setbold#1{\boldfont #1}
\def\setdynamic#1{\dynfont #1}
\def\setfinger#1{\fingerfont #1}
\def\setlarge#1{\largefont #1}
\def\setnumber#1{\fetanummer #1}

% the interline symbol. Redefine to remove it.
\def\defaultlineseparator{\vbox{\mussepline\vskip -5pt\mussepline}}
\def\lineseparator{\defaultlineseparator}
\def\beauty{%
        \par\vskip 10pt plus 30pt minus 10pt\par
        \hskip -5pt\lineseparator
        \par\vskip 10pt plus 30pt minus 10pt\par
}


\def\interscoreline{\vskip 16pt}
        
%%%%%%%%%%%%%%%%%%%%%%%%%%%%%%%%%
% big fat marks, if errors are detected.
\def\columnerrormark{\placebox{-5pt}{0pt}{\bf C!}}
\def\scorelineerrormark{\placebox{0pt}{-10pt}{\bf L!}}
\def\errormark{{\bf E!}}
\def\unknown{{\bf u}}

\def\postheader{}
\ifundefined{documentclass}
        \footline={\ifnum\pageno=1
        {\smalltextfont\mudelacopyright\hfil \LilyIdString
        }\else{\hfil\the\pageno\hfil}\fi
}\else
        %% FIXME
        \def\ps@plain{
                \renewcommand{\@oddhead}{}%
                \renewcommand{\@evenfoot}{}%
                \renewcommand{\@evenhead}{}%
                \renewcommand{\@oddfoot}{\ifnum\thepage=1
        {\hfil \LilyIdString
        }\else{foo\hfil\the\pageno\hfil}\fi}}
        \def\ps@empty{
                \renewcommand{\@oddhead}{}%
                \renewcommand{\@evenfoot}{}%
                \renewcommand{\@evenhead}{}%
                \renewcommand{\@oddfoot}{\ifnum\thepage=1
        {\hfil \LilyIdString
        }\else{foo\hfil\the\pageno\hfil}\fi}}
\fi


% debugging stuff:
% \vbox to 0pt{\vskip .5cm \hruler{48}{3pt}\vss}
