%%
%% include file for LilyPond
%%
%% this file defines various macros to accomodate lilypond output

% TeXbook ex 7.7
\def\ifundefined#1{\expandafter\ifx\csname#1\endcsname\relax}

\def\SkipLilydefs{\endinput}
\ifundefined{EndLilyPondOutput}
        \def\EndLilyPondOutput{\csname bye\endcsname}
        \def\SkipLilydefs{}
\fi
\SkipLilydefs
% should use \endinput

        \def\mdef#1#2{\def#1{\mchar{#2}}}
\def\mchar#1{\musicfnt\char#1}
\def\rationalmultiply#1*#2/#3{\multiply #1 by #2 \divide #1 by #3}
\def\maccentraise#1#2{\dimen0=\noteheight
        \rationalmultiply\dimen0*#2%
        \raise\dimen0\hbox{#1}}
\def\maccentdef#1#2#3{\def#1{\maccentraise{\mchar{#2}}{#3}}}
\def\vertcenter#1{\vbox to 0pt{\vss #1\vss}}

\def\mathdef#1#2{\def#1{\mathchar{#2}}}
\def\mathchar#1{\mathfont\char#1}

\def\topalign#1{\vbox to 0pt{#1\vss}}
\def\botalign#1{\vbox to 0pt{\vss #1}}

\def\centeralign#1{\hbox to 0pt{\hss#1\hss}}
\def\leftalign#1{\hbox to 0pt{#1\hss}}
\def\rightalign#1{\hbox to 0pt{\hss#1}}


%% musix defs
\parindent=0pt
\newdimen\smallspace
\newdimen\interlinedist
        \newcount\n
        \newdimen\balkhoog
        \newdimen\notewidth
        \newdimen\noteheight
        \newdimen\notewidthhalf
        \newdimen\notewidthdouble
        \newdimen\staffrulethickness
        \newdimen\interstaffrule
\newdimen\balkhalf
\def\musixtwentydefs{
        \font\textfont=cmr10
        \font\meterfont=cmbx12
        \font\slurufont=xslu20
        \font\slurdfont=xsld20
        \font\slurhfont=xslz20
        \font\hslurufont=xslhu20
        \font\hslurdfont=xslhd20
        \font\hslurhfont=xslhz20
        \font\musicfnt=musix20
        \font\musicdraw=musixsps
        \font\italicfont=cmti10
        \font\dynfont=cmbxti10 scaled \magstep1
        \font\mathfont=cmsy10
        \balkhoog=20pt
        \notewidth=6pt
        \noteheight=5pt
        \staffrulethickness=0.4pt
        \musixcalc
}


\def\musixsixteendefs{
        \font\textfont=cmr8
        \font\meterfont=cmbx12
        \font\italicfont=cmti8
        \font\slurufont=xslu16
        \font\slurdfont=xsld16
        \font\slurhfont=xslz20 % sigh
        \font\hslurufont=xslhu16
        \font\hslurdfont=xslhd16
        \font\hslurhfont=xslhz20
        \font\musicfnt=musix16
        \font\dynfont=cmbxti12
        \font\musicdraw=musixsps
        \font\mathfont=cmsy8
        \balkhoog=16pt
        \staffrulethickness=0.4pt
        \notewidth=5pt
        \noteheight=4pt
        \musixcalc
}

\def\musixcalc{
        \interlinedist=\fontdimen5\musicfnt
        \smallspace=.3\interlinedist
        \interstaffrule=\balkhoog
        \advance\interstaffrule by-\staffrulethickness
        \divide\interstaffrule by 4
        \advance\interstaffrule by -\staffrulethickness
        \notewidthhalf=\notewidth
        \notewidthdouble=\notewidth
        \rationalmultiply \notewidthhalf*1/2
        \multiply \notewidthdouble by 2
               \balkhalf=\balkhoog
 \rationalmultiply\balkhalf*1/2 
}
% \def\dyn{\italicfont}
\def\dyn{\dynfont}
\def\kdynf{\dynfont f\kern-.1ex}
\def\kdynm{\dynfont m\kern-.15ex}
\def\kdynp{\dynfont p\kern-.15ex}


\def\dynp{\dynfont p}
\def\dynpp{\dynp\kdynp}
\def\dynppp{\dynpp\kdynp}
\def\dynmp{\dynfont m\kdynp}
\def\dynmf{\dynfont m\kdynf}
\def\dynf{\dynfont f}
\def\dynff{\dynf\kdynf}
\def\dynfff{\dynff\kdynf}

\def\slurcharh#1{{\slurhfont\char#1}}
\def\slurcharu#1{{\slurufont\char#1}}
\def\slurchard#1{{\slurdfont\char#1}}
\def\hslurcharh#1{{\hslurhfont\char#1}}
\def\hslurcharu#1{{\hslurufont\char#1}}
\def\hslurchard#1{{\hslurdfont\char#1}}

\mdef\quartball{'007}
\mdef\halfball{'010}
\mdef\wholeball{'011}
\mdef\halfrest{'074}
\mdef\wholerest{'075}
\mdef\quartrest{62}
\mdef\eighthrest{63}
\mdef\sixteenthrest{64}
\mdef\thirtysecondrest{65}
\mdef\sharp{52}
\mdef\flat{50}
\mdef\natural{54}
\mdef\sharpsharp{53}
\mdef\flatflat{51}

\mdef\singledot{'00}
\mdef\doubledot{'01}
\mdef\tripledot{'02}
\mdef\mussepline{155}

\mdef\violinclef{71}
\mdef\bassclef{73}
\mdef\altoclef{75}
\mdef\cviolinclef{72}
\mdef\cbassclef{74}
\mdef\caltoclef{76}

\mdef\deigthflag{45}
\mdef\dsixteenthflag{46}
\mdef\dthirtysecondflag{47}

% pointing up
\mdef\ueigthflag{40}
\mdef\usixteenthflag{41}
\mdef\uthirtysecondflag{42}

\maccentdef\repeatcolon{55}{2/1}
\def\emptybar{}

\def\thinbar{\vrule height\balkhoog}
\def\thickbar{\vrule height\balkhoog width 2\smallspace}
\def\maatstreep{\thinbar}
%? what-s wrong with rightalign?
\def\finishbar{\rightalign{\thinbar\kern\smallspace\thickbar}}
%%% \def\finishbar{\hss\rightalign{\thinbar\kern\smallspace\thickbar}}
% \def\repeatstopbar{\rightalign{\repeatcolon\hskip2\smallspace\thinbar\hskip\smallspace\thickbar}}
\def\repeatstopbar{\hss\rightalign{\repeatcolon\hskip2\smallspace\thinbar\hskip\smallspace\thickbar}}
\def\repeatstartbar{\hbox{\thickbar\kern\smallspace\thinbar\kern2\smallspace\repeatcolon}}
\def\repeatstopstart{\hbox{\repeatcolon\kern2\smallspace\thinbar\kern\smallspace\thickbar\kern\smallspace\thickbar\kern\smallspace\thinbar\kern2\smallspace\repeatcolon}}
\def\doublebar{\hbox{\thinbar\hskip\smallspace\thinbar}}

%compatability
\def\repeatbar{\repeatstopbar}
\def\startrepeat{\repeatstartbar}
\def\repeatbarstartrepeat{\repeatstopstart}

\def\generalmeter#1#2{\botalign{\vbox to\balkhalf{\vss \meterfont#1}%
        \nointerlineskip
        \vbox to \balkhalf{\vss\meterfont #2}}}
\def\defaultlineseparator{\vbox{\mussepline\vskip -5pt\mussepline}}
\def\lineseparator{\defaultlineseparator}
\def\beauty{%
        \par\vskip 10pt plus 30pt minus 10pt\par
        \hskip -5pt\lineseparator
        \par\vskip 10pt plus 30pt minus 10pt\par
}

\def\interstaffline{%
        \vskip 10pt
}
\def\ugly{\nointerlineskip\par
\vskip 40pt\par\vbox{\leftalign{\vrule width30pt height1pt}}\par\vskip 40pt
}
\def\interscoreline{\beauty}


\def\lines#1#2{%
 \vbox{\kern-\interstaffrule
        \n=0\nointerlineskip%
        \loop\ifnum\n<#1\advance\n by1%
                \kern\interstaffrule
                \hrule height \staffrulethickness width#2
                \repeat
        }}

\def\toplines#1{ % why space needed here?
        \topalign{\hbox{\kern-\notewidth\lines{#1}{\notewidthdouble}}}}
\def\botlines#1{ % idem ditto
        \botalign{\hbox{\kern-\notewidth\lines{#1}{\notewidthdouble}}}}

%
% a staffsymbol with #1 lines, width #2
% bottom at baseline
\def\linestafsym#1#2{\leftalign{\botalign{\lines{#1}{#2}}}}

\def\eigthflag{\topalign{\ueigthflag}}
\def\sixteenthflag{\topalign{\usixteenthflag}}
\def\thirtysecondflag{\topalign{\uthirtysecondflag}}

\def\cquartrest{\vertcenter\quartrest}
\def\ceighthrest{\vertcenter\eighthrest}
\def\csixteenthrest{\vertcenter\sixteenthrest}
\def\cthirtysecondrest{\vertcenter\thirtysecondrest}

\def\lsingledot{\kern-\notewidth\singledot}
\def\ldoubledot{\kern-\notewidth\doubledot}
\def\ltripledot{\kern-\notewidth\tripledot}

\maccentdef\sforzato{30}{-3/2}
\maccentdef\marcato{20}{-1/1}
\maccentdef\imarcato{21}{1/1}
\maccentdef\staccato{24}{-1/3}
\maccentdef\istaccato{25}{1/3}
\maccentdef\staccatissimo{28}{-5/5}
\maccentdef\istaccatissimo{29}{2/5}
\maccentdef\portato{18}{-6/5}
\maccentdef\iportato{19}{3/5}
\maccentdef\tenuto{26}{-1/1}
\maccentdef\itenuto{27}{1/1}
\maccentdef\fermata{80}{-1/1}
\maccentdef\ifermata{81}{1/1}

\mdef\spicato{28}
\mdef\ispicato{29}
\mdef\upbow{23}
\mdef\downbow{22}

\mathdef\cup{91} % \cup
\mathdef\wedge{94} % \wedge
\mathdef\striepke{0} % heu?

%% custom characters --- this should go: add to (meta!) font
\def\myheel{\kern-.5ex\vbox{\cup}\kern-.5ex}
\def\mytoe{\kern-.5ex\vbox{\wedge}\kern-.5ex}
\def\mystriepke{\kern-1.1ex\vbox{\hbox{\kern-.05em\striepke}}\kern-1.1ex}

\def\heel{\vbox{\myheel}}
\def\toe{\vbox{\mytoe}}
\def\backorfront{\mystriepke}

\def\bheel{\vbox{\myheel\mystriepke}}
\def\btoe{\vbox{\mytoe\mystriepke}}
\def\fheel{\vbox{\mystriepke\myheel}}
\def\ftoe{\vbox{\mystriepke\mytoe}}
\def\heeltoe{\vbox{\myheel\mytoe}}
\def\toeheel{\vbox{\mytoe\myheel}}
%%

\def\stem#1#2{\vrule height#2 depth-#1}

\def\placebox#1#2#3{%
        \botalign{\hbox{\raise #1\leftalign{\kern #2{}#3}}}%
}


\def\beamslope#1#2{{\count0=#2\advance\count0 by#1\musicfnt\char\count0}}
\def\rulesym#1#2{\vrule height #1 width #2}
\def\crescendosym#1{\count0=84\advance\count0 by #1\musicdraw\char\count0}
\def\decrescendosym#1{\count0=116\advance\count0 by #1\musicdraw\char\count0}
\def\settext#1{\textfont #1}
\def\setitalic#1{\italicfont #1}
\def\setdynamic#1{\dynfont #1}

        

