\documentclass[a4paper]{article}
\begin{document}

In a lilypond-book document, you can freely mix music and text. For
example:
\begin{lilypond}
  \score { \notes \relative c' {
     c2 g'2 \times 2/3 { f8 e d } c'2 g4
  } }
\end{lilypond}
Notice that the music line length matches the margin settings of the
document.

If you have no \verb+\score+ block in the fragment,
\texttt{lilypond-book} will supply one:

\begin{lilypond}
  c'4
\end{lilypond}

In the example you see here, a number of things happened: a
\verb+\score+ block was added, and the line width was set to natural
length. You can specify many more options using  \LaTeX style options
in brackets:

\begin[verbatim,11pt,singleline,
  fragment,relative,intertext="hi there!"]{lilypond}
  c'4 f bes es
\end{lilypond}

\texttt{verbatim} also shows the lilypond code, \texttt{11pt} selects
the default music size, \texttt{fragment} adds a score block,
\texttt{relative} uses relative mode for the fragment, and
\texttt{intertext} specifies what to print between the
\texttt{verbatim} code and the music.

If you include large examples into the text, it may be more convenient
to put the example in a separate file:

\lilypondfile[printfilename]{sammartini.ly}

The \texttt{printfilename} option adds the file name to the output.

\end{document}
